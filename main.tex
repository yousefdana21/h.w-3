\documentclass{article}
\usepackage{amsmath} % for mathematical formulas
\usepackage{amssymb} % for mathematical symbols

\title{Exploring the Wonders of Mathematics}
\author{Yousef dana}
\date{April 30, 2024}

\begin{document}

\maketitle

\section*{Abstract}
Join us on a journey
through the wonders of
mathematics, from ba-
sic equations to complex
concepts.

\section*{Equations}

Marvel at the simplicity yet pro- fundity of Einstein’s equation:
\[E = mc^2\]

Observe the elegance of the Pythagorean theorem:
\[a^2 + b^2 = c^2\]

Witness the beauty of mathe- matical functions:
\[f(x) = x^2\]
\[g(x) = \frac{1}{x}\]
\[h(x) = \sin(x)\]

Marvel at the harmony of lin-ear equations:\[2x - 5y = 8\]
\[3x + 9y = 3\]

Behold the elegance of integra-tion:
\[\int_0^\infty e^{-x^2} \, dx = \sqrt{\frac{\pi}{2}}\]

Marvel at the simplicity of fractions:
\[\frac{3}{4}, \alpha, \beta, \gamma, \Delta, \Sigma\]

\[\int_0^\infty e^{-x^2} \cos(2x) \sin^2(x) \, dx = \sqrt{\frac{\pi}{8}}\]

Observe the power of matrices and arrays:
\[\begin{matrix}1 & 2 \\3 & 4\end{matrix}\]
\[\begin{array}{ccc}
a & b & c \\
d & e & f \\
g & h & i
\end{array}\]

Witness the versatility of sub-scripts and superscripts:
\[x_1, x^2\]

\section*{Citations and References}

In our quest for knowledge, it is essential to acknowledge the con-tributions of others.

\section*{Citation}
Explore the realms of machine learning with Mitchell’s seminal work [1].

\section*{Bibliography}
\section*{References}
[1] Mitchell, T. M. (2007). Ma-chine Learning. McGraw-Hill.Pause for a moment of reflec-tion with a footnote.1
\footnote{This is an example of a footnote.}

\end{document}
